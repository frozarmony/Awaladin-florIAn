\documentclass[]{article}

\usepackage{subfiles}
\usepackage[utf8]{inputenc}
\usepackage[francais]{babel}
\usepackage{amsmath,amsfonts,a4,fourier,graphicx}
   \voffset=-1in
   \hoffset=-1in
   \topmargin=1.5cm
   \headheight=0cm
   \headsep=0cm
   \setlength{\paperheight}{29.7cm}%
   \setlength{\paperwidth}{21cm}%
   \setlength{\oddsidemargin}{2.5cm}%
   \setlength{\evensidemargin}{2.5cm}%
   \setlength{\marginparsep}{0cm}%
   \setlength{\marginparwidth}{0cm}%
   \setlength{\footskip}{1cm}% 
   \setlength{\textheight}{24cm}%
   \setlength{\textwidth}{16cm}%
   \setlength{\parskip}{2ex}%

\usepackage{etex}
\usepackage{m-pictex,m-ch-en}

\usepackage{color}
\usepackage{listings}


\definecolor{mygreen}{rgb}{0,0.6,0}
\definecolor{mygray}{rgb}{0.5,0.5,0.5}
\definecolor{mymauve}{rgb}{0.58,0,0.82}
\definecolor{lightGray}{rgb}{0.99, 0.99, 0.99}

\lstset{ %
  backgroundcolor=\color{lightGray},   % choose the background color; you must add \usepackage{color} or \usepackage{xcolor}
  basicstyle=\footnotesize,        % the size of the fonts that are used for the code
  breakatwhitespace=false,         % sets if automatic breaks should only happen at whitespace
  breaklines=true,                 % sets automatic line breaking
  captionpos=b,                    % sets the caption-position to bottom
  commentstyle=\color{mygreen},    % comment style
  deletekeywords={...},            % if you want to delete keywords from the given language
  escapeinside={\%*}{*)},          % if you want to add LaTeX within your code
  extendedchars=false,              % lets you use non-ASCII characters; for 8-bits encodings only, does not work with UTF-8
  frame=single,                    % adds a frame around the code
  keepspaces=true,                 % keeps spaces in text, useful for keeping indentation of code (possibly needs columns=flexible)
  keywordstyle=\color{blue},       % keyword style
  language=Octave,                 % the language of the code
  morekeywords={*,...},            % if you want to add more keywords to the set
  numbers=left,                    % where to put the line-numbers; possible values are (none, left, right)
  numbersep=5pt,                   % how far the line-numbers are from the code
  numberstyle=\tiny\color{mygray}, % the style that is used for the line-numbers
  rulecolor=\color{black},         % if not set, the frame-color may be changed on line-breaks within not-black text (e.g. comments (green here))
  showspaces=false,                % show spaces everywhere adding particular underscores; it overrides 'showstringspaces'
  showstringspaces=false,          % underline spaces within strings only
  showtabs=false,                  % show tabs within strings adding particular underscores
  stepnumber=1,                    % the step between two line-numbers. If it's 1, each line will be numbered
  stringstyle=\color{mymauve},     % string literal style
  tabsize=2,                       % sets default tabsize to 2 spaces
  title=\lstname                   % show the filename of files included with \lstinputlisting; also try caption instead of title
}


%opening
\title{IA02 - Rapport}
\author{BAUNE Florian - TALEB Aladin}
\date{}

\begin{document}

\maketitle

\begin{abstract}

\end{abstract}

\section{Présentation du jeu}

L'Awale dans sa version de base est un jeu qui se joue à 2 joueurs. Chaque joueur possède un plateau, avec 6 trous qui contiennent un certain nombre de graines.

On définit la situation initiale comme suit : 
\begin{itemize}
\item Les scores des deux joueurs à 0
\item 4 graines dans chaque trou de chaque plateau
\end{itemize}

Les joueurs jouent au tour à tour. Voici le déroulement d'un tour : 

\begin{enumerate}
\item Prise de toutes les graines d'un des trous non vide dans le plateau du joueur et distribution des graines dans le sens anti-horaire à partir du trou suivant le trou source.
\subitem Une prise est valide si l'adversaire a au moins une graine dans son plateau à la fin de la distribution. Si aucune prise n'est possible, chaque joueur récole le contenu des graines de son plateau.
\item Récolte des graines à condition que :
	\begin{itemize}
	\item La dernière graine distribuée soit dans le plateau du joueur adverse
	\item Le dernier trou rempli contient 2 ou 3 graines. 
	\subitem Dans ces cas, le joueur récolte les graines dans le trou, ainsi que les graines de trous précédents s'ils respectent également ces conditions
	\item Le plateau adverse contient au moins une graine à la fin de la récolte. Si ce n'est pas le cas, on annule toute éventuelle récolte, sans annuler la distribution des graines.
	\end{itemize}
	\subitem Les graines récoltées sont ajoutées au score du joueur.
\end{enumerate}

Ces tours s'effectuent jusqu'à ce que :
\begin{itemize}
\item L'un des joueurs gagne, c'est à dire que son score est strictement supérieur à la moitié des graines du plateau
\item Le plateau n'ai plus de graine, ce qui généralement signifie un jeu nul
\item Le jeu soit cyclique, dans ce cas, les joueurs récupèrent les graines de leur camps. Dans les règles officielles, la cyclicité doit être observée et validée par les deux joueurs. Pour notre programme, nous définissons la cyclicité de manière simple et intuitive, quitte à être restrictive : si le jeu "boucle", donc que nous rencntrons un état déjà joué, alors le jeu est cyclique.
\end{itemize}



\section{Hypothèses \& Manière de travailler}

	Lors de notre premier TP de projet, nous avons cherché à comprendre les règles de l'awalé à la fois par la théorie et par la pratique (il faut bien s'amuser un peu).
Puis très vite, notre chargée de TD nous a proposé quelques pistes pour nous aider à démarer. Ces pistes se présentées sous la forme de prototypes de prédicats sensés apporter une première pierre à notre édifice.
Simplement, nous manquions de recule sur l'architecture globale sous-jacente à ces prototypes. C'est donc dans une optique de clarté et de cohérence que nous avons décidé de faire table raze et de trouver une architecture par nos propres moyens.

\subsection{Organisation du travail}

	De manière générale, nous avons adopté la même méthode pour chaque point critique du projet.
Ainsi, nous nous réunissions autour d'un tableau afin de discuter de chacunes des difficultés. Puis, au fur et à mesure de nos échanges, nous construisions l'architecture et décidions des meilleurs solutions à apporter à nos problèmes.
Enfin, lorsqu'un point de conception était suffisament détaillé, nous nous répartissions les prédicats à développer et nous donnions rendez-vous quelques jours plus tard pour la fusion et les tests de nos codes respectifs.
En somme dans ce projet, nous avons conçu par le haut pour mieux développer et tester par le bas.

Pour ma part (Florian), j'ai trouvé cette expérience très enrichissante, car elle fut un très bon exemple de synergie entre créativité et efficacité.

\subsection{Cahier des charges}

\section{Architecture}

TODO Flo

\subsection{Structures de données}

TODO Flo

\subsection{Fonction principale}



\subsection{Initialisation}

\subsection{Boucle principale}

\subsection{Tour de jeu}



\subsection{Actions}

\subsubsection{Distribuer graines}

TODO Flo

\subsubsection{Récolter graines}

\subsubsection{Vider les plateaux}



\subsection{IA}

\subsubsection{Génération d'arbre}

TODO Flo

\subsubsection{Parcours d'arbre}

\section{Modularité \& Evolutivité}

TODO Flo
	- état initiale modulable
	- variation sur le nombre de champs, et le nombre de graine de départ
	- ajout autre type de joueur et/ou IA

\section{Conclusion}



\end{document}
